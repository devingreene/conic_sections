% vim: se spell tw=70:
\documentclass[12pt]{article}

\usepackage{amsmath}

%macros
\newcommand{\eps}{\epsilon}

\title{Conic sections are in fact conic sections}
\author{Devin Greene}
\date{}
\begin{document}
\maketitle

In this article, I will show that every non-degenerate quadratic
planar curve is congruent to the intersection of a plane in three
dimensional space with the double-cone.  In what follows, when I 
refer to objects in the plane, I use lower-case $x$ and $y$, and
when referring to objects in space, I use upper-case $X$, $Y$, and
$Z$.

A double-cone aligned with the coordinate axes has the form 

\begin{equation} \label{canon} \eps_x X^2 + \eps_y Y^2 + \eps_z
Z^3 = 0\end{equation}
where
$$\begin{matrix}(\eps_x,\eps_y,\eps_z) & \in  &
\{ &(1,1,-1),&(1,-1,1),&(-1,1,1), &\\
&&&(1,-1,-1),&(-1,1,-1),&(-1,-1,1) &\}\end{matrix}$$
In each demonstration below, we will show that a curve is congruent to
the intersection of the plane $Z = 1$ and a surface congruent to one
of the forms (\ref{canon}).

Note that the double-cone defined in (\ref{canon}) is invariant under
uniform scaling by a factor.  Thus for each quadratic curve it
suffices to prove our assertion for any {\it similar} curve.

Begin by noting that via translation and orthogonal transformation on
the plane, any non-degenerate quadratic curve can be written in one of
the following ways:

\begin{enumerate}
\item {\it (ellipse)} $a^2x^2 + b^2y^2 - 1 = 0$,
\item {\it (hyperbola)} $-a^2x^2 + b^2y^2 - 1 = 0$, or
\item {\it (parabola)} $y - a^2x^2 = 0$
\end{enumerate}
where $a,b \neq 0$ are constants.

\medskip

\noindent {\it Case 1}: As noted above, it suffices to prove this for
any similar curve, thus\footnote{
In the equation $a^2x^2 + b^2y^2 - 1 = 0$, if $a^2 < b^2$, then swap
variables.  Let $c^2 = (b/a)^2$, then $x^2 + c^2 y^2 - 1/a^2 = 0$.
Choose $\sigma$ such that $1/c^2 = 1/a^2\sigma^2$.  Then
$(\sigma x)^2
+ c^2 (\sigma y)^2 - 1/a^2 = 0 \implies x^2  + c^2 y^2 - 1/a^2\sigma^2
= 0 \implies x^2 + c^2 y^2 - 1/c^2 = 0$
}
we may assume the
equation has the form $x^2 + c^2 y^2 - 1/c^2 = 0$, where $|c| < 1$. Translate the curve in the $y$ direction by $-t = -\sqrt{1/c^4 - 1}$,
obtaining $x^2 + c^2 (y + t)^2 - 1/c^2 = 0$.  This curve is congruent to
the intersection of the surface $X^2 + c^2 (Y + tZ)^2 - Z^2/c^2 = 0$ and
the plane $Z = 1$ in 3-space.  Hence it suffices to show that $ X^2 +
c^2 (Y + tZ)^2 - Z^2/c^2 = 0$ is congruent to a double cone.

To this end, expand the left-hand side to get

$$X^2 + c^2 Y^2 + 2c^2t YZ - (1/c^2 - c^2t^2)Z^2$$
This corresponds to the symmetric form given by the matrix 

\begin{equation}\label{form}\left(
\begin{matrix} 
1 & 0 & 0 \\
0 & c^2 & c^2t \\
0 & c^2 t& -1/c^2 + c^2t^2\\
\end{matrix}
\right)
\end{equation} 
From the definition of $t$ above, this becomes
$$\left(
\begin{matrix}
1 & 0 & 0 \\
0 & c^2 & c^2 t\\
0 & c^2 t& -c^2\\
\end{matrix}
\right)$$
Thus the characteristic polynomial of the lower right sub-matrix is

$$( x^2 - c^2)(x^2 + c^2) -  c^4 t^2 = x^2 - c^4(1 + t^2)$$
which, again from the definition of $t$, is

\begin{equation}\label{eigen} x^2  -1 = (x - 1)(x + 1)\end{equation}
Since (\ref{form}) is symmetric, (\ref{eigen}) shows that (\ref{form}) is
equivalent via orthogonal transformation to

$$\left(\begin{matrix}1 & 0 & 0 \\
0 & 1 & 0\\
0 & 0 & -1\end{matrix}\right).$$
Thus the equation of the surface is congruent to
$$X^2 + Y^2 - Z^2 = 0,$$
which is a double cone.

\medskip

\noindent {\it Case 2:} This case is similar to Case 1, ending in the
double-cone $-X^2 + Y^2 - Z^2 = 0$.

\medskip

\noindent {\it Case 3:} First note that by scaling 
we can eliminate the $a^2$ and introduce a factor of $2$.  Namely,
$y/2a^2 - a^2(x/2a^2)^2  = 0\implies 2y
- x^2 = 0$.  This curve is congruent to the intersection of
the surface $2YZ - X^2 = 0$ and the plane $Z = 1$.  Hence it suffices
to show that $2YZ - X^2 = 0$ is congruent to a double cone.  
Here we simply take the orthogonal transformation on the
$YZ$-plane

$$\left(\begin{matrix}
Y\\
Z\end{matrix}\right) \mapsto
\left(\begin{matrix}
\frac{1}{\sqrt{2}} & \frac{1}{\sqrt{2}} \\
-\frac{1}{\sqrt{2}} & \frac{1}{\sqrt{2}}\end{matrix}
\right) \left(\begin{matrix}
Y\\Z\end{matrix}\right)$$
which converts the term $2YZ$ to $Y^2 - Z^2$.  Hence the surface is
congruent to 

$$Y^2 - Z^2 - X^2 = 0$$
which is a double-cone.

\end{document}
