% vim: se spell tw=70:
\documentclass[12pt]{article}

\usepackage{fullpage}
\usepackage{listings}
\usepackage{color,xcolor}
\usepackage{amsmath}
\usepackage{graphicx}
\usepackage{tikz}

\renewcommand{\t}{\texttt}
\newcommand{\bs}{\bigskip}
\newcommand{\e}{\text{e}}
\newcommand{\p}{\partial}
\renewcommand{\d}[2]{\frac{d#1}{d#2}}

\begin{document}

In this article, I will show that every non-degenerate quadratic
planar curve is congruent to the intersection of a plane in three
dimensional space with the double-cone.  In what follows, when we are
referring to objects in the plane, we use lower-case $x$ and $y$, and
when referring to objects in space, we use upper-case $X$, $Y$, and
$Z$.

We know that via rotation, translation, or reflection in the plane, every
quadratic planar curve has one of the following forms:

\begin{enumerate}
\item {\it (ellipse)} $a^2x^2 + b^2y^2 - 1 = 0$,
\item {\it (hyperbola)} $a^2x^2 - b^2y^2 - 1 = 0$, or
\item {\it (parabola)} $y - a^2x^2 = 0$
\end{enumerate}
where $a,b$ are constants.

\medskip

\noindent {\it Case 1}: The double-cone is invariant under scaling uniformly along all three spatial axes, thus without loss of generality we may assume the
equation has the form $x^2 + c^2 y^2 - 1/c^2 = 0$, where $|c| < 1$. \footnote{
In the equation $a^2x^2 + b^2y^2 - 1 = 0$, if $a^2 < b^2$, then swap
variables.  Let $c^2 = (b/a)^2$, then $x^2 + c^2 y^2 - 1/a^2 = 0$.
Choose $\sigma$ such that $1/c^2 = 1/a^2\sigma^2$.  Then
$(\sigma x)^2
+ c^2 (\sigma y)^2 - 1/a^2 = 0 \implies x^2  + c^2 y^2 - 1/a^2\sigma^2
= 0 \implies x^2 + c^2 y^2 - 1/c^2 = 0$
}
Translate the curve in the $y$ direction by $-t = -\sqrt{1/c^4 - 1}$,
obtaining $x^2 + c^2 (y + t)^2 - 1/c^2 = 0$.  This curve is congruent to
the intersection of the surface $X^2 + c^2 (Y + tZ)^2 - Z^2/c^2 = 0$ and
the plane $Z = 1$ in 3-space.  Hence it suffices to show that $ X^2 +
c^2 (Y + tZ)^2 - Z^2/c^2 = 0$ is congruent to a double cone.

To this end, expand the left-hand side to get

$$X^2 + c^2 Y^2 + 2c^2t YZ - (1/c^2 - c^2t^2)Z^2$$
This corresponds to the symmetric form given by the matrix 

\begin{equation}\label{form}\left(
\begin{matrix} 
1 & 0 & 0 \\
0 & c^2 & c^2t \\
0 & c^2 t& -1/c^2 + c^2t^2\\
\end{matrix}
\right)
\end{equation} 
From the definition of $t$ above, this becomes
$$\left(
\begin{matrix}
1 & 0 & 0 \\
0 & c^2 & c^2 t\\
0 & c^2 t& -c^2\\
\end{matrix}
\right)$$
Thus the characteristic polynomial of the lower right submatrix is

$$( x^2 - c^2)(x^2 + c^2) -  c^4 t^2 = x^2 - c^4(1 + t^2)$$
which, again from the definition of $t$, is

\begin{equation}\label{eigen} x^2  -1 = (x - 1)(x + 1)\end{equation}
Since (\ref{form}) is symmetric, (\ref{eigen}) shows that  can thus
diagonalize it via an orthogonal transformation to

$$\left(\begin{matrix}1 & 0 & 0 \\
0 & 1 & 0\\
0 & 0 & -1\end{matrix}\right).$$
Thus the equation of the surface is
$$X^2 + Y^2 - Z^2 = 0,$$
which is a double cone.

\medskip

\noindent {\it Case 2:} This case is similar to Case 1 and is left as an
exercise.

\medskip

\noindent {\it Case 3:} First note that by scaling in both dimensions
we can eliminate the $a^2$ and introduce a factor of $2$.  Namely,
$y/2a^2 - a^2(x/2a^2)^2  = 0\implies 2y
- x^2 = 0$.  The curve is congruent to the intersection of
the surface $2YZ - X^2 = 0$ and the plane $Z = 1$.  Hence it suffices
to show that $2YZ - X^2 = 0$ is congruent to a double cone.  This is
trivial as we simply take the orthogonal transformation on the
$YZ$-plane

$$\left(\begin{matrix}
Y\\
Z\end{matrix}\right) \mapsto
\left(\begin{matrix}
\frac{1}{\sqrt{2}} & \frac{1}{\sqrt{2}} \\
-\frac{1}{\sqrt{2}} & \frac{1}{\sqrt{2}}\end{matrix}
\right) \left(\begin{matrix}
Y\\Z\end{matrix}\right)$$
which converts the term $2YZ$ to $Y^2 - Z^2$.  Hence the surface is
congruent to 

$$Y^2 - Z^2 - X^2 = 0$$
which is a double-cone.

\end{document}
